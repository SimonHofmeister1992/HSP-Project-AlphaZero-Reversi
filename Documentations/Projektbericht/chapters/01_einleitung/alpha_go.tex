\subsection{AlphaGo}
\label{secAlphaGo}
AlphaGo ist ein Computerprogramm, das das Brettspiel \glqq{}Go\grqq{} spielt \cite{BBC_NewsGo}. Es wurde von \textit{DeepMind Technologies} entwickelt, das später von \textit{Google} übernommen wurde. AlphaGo hatte drei weitaus mächtigere Nachfolger, genannt AlphaGo Master, AlphaGo Zero und AlphaZero \cite{DeepMind}.

Im März 2016 schlug AlphaGo Lee Sedol in einem Fünf-Spiel-Match. Das war das erste Mal, dass ein Computer-Go-Programm einen 9-Dan-Profi ohne Handicap besiegte \cite{AlphaGoMatch}. Obwohl es im vierten Spiel gegen Lee Sedol verlor, resultierte ein Endergebnis von 4 zu 1 zu Gunsten von AlphaGo. In Anerkennung des Sieges wurde AlphaGo von der Korea Baduk Association mit einem Ehren-9-Dan ausgezeichnet \cite{AlphaGoDan}. Der Vorsprung und das Herausforderungsspiel mit Lee Sedol wurden in einem Dokumentarfilm mit dem Titel AlphaGo \cite{AlphaGoFilm} unter der Regie von Greg Kohs dokumentiert. Er wurde am 22. Dezember 2016 von Science als einer der Durchbrüche des Jahres betitelt \cite{Science2016}.

AlphaGo und seine Nachfolger verwenden einen Monte Carlo Baumsuch-Algorithmus, um Spielzüge auf der Grundlage von Wissen zu finden, das zuvor durch maschinelles Lernen \glqq{}gelernt\grqq{} wurde. Dies geschieht insbesondere durch ein künstliches neuronales Netz, das mithilfe von Spieldurchläufen ausführlich trainiert wurde \cite{Silver_2016}. Ein neuronales Netz wird trainiert, um AlphaGo's eigene Zugauswahl, sowie die Gewinner der Partien vorherzusagen. Dieses neuronale Netz verbessert die Stärke der Baumsuche, was zu einer höheren Qualität der Zugauswahl und einem stärkeren Selbstspiel in der nächsten Iteration führt.

Hinsichtlich des Spielstils von AlphaGo ist folgendes festzuhalten. Im Spiel gegen einen Top-Go-Spieler hat das künstliche Intelligenzprogramm AlphaGo die Kommentatoren mit Zügen verwirrt, die oft als \glqq{}schön\grqq{} beschrieben wurden, aber nicht in den üblichen menschlichen Spielstil passen \cite{Ribeiro2016}.

Howard Yu, Professor für strategisches Management und Innovation an der IMD Business School meinte, dass AlphaGo  eine Maschine darstellt, die nicht nur denkt, sondern auch lernen und Strategien entwickeln kann \cite{Ribeiro2016}.

Toby Manning, der Match-Schiedsrichter für AlphaGo vs. Fan Hui, hat den Stil des Programms als \glqq{}konservativ\grqq{} beschrieben \cite{Gibney2016}. Der Spielstil von AlphaGo begünstigt stark die größere Wahrscheinlichkeit, mit weniger Punkten zu gewinnen, gegenüber der geringeren Wahrscheinlichkeit, mit mehr Punkten zu gewinnen \cite{Ribeiro2016}. Seine Strategie, die Gewinnwahrscheinlichkeit zu maximieren, unterscheidet sich von dem, wozu menschliche Spieler neigen, nämlich territoriale Gewinne zu maximieren \cite{Chouard2016}.

