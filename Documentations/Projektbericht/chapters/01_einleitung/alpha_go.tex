\subsection{AlphaGo}
AlphaGo ist ein Computerprogramm, das das Brettspiel Go spielt\cite{BBC_NewsGo}. Es wurde von DeepMind Technologies entwickelt, das später von Google übernommen wurde. AlphaGo hatte drei weitaus mächtigere Nachfolger, genannt AlphaGo Master, AlphaGo Zero und AlphaZero \cite{DeepMind}.

Im März 2016 schlug AlphaGo Lee Sedol in einem Fünf-Spiel-Match, das erste Mal, dass ein Computer-Go-Programm einen 9-Dan-Profi ohne Handicap besiegte \cite{AlphaGoMatch}. Obwohl es im vierten Spiel gegen Lee Sedol verlor, trat Lee im Endspiel zurück und gab im Endergebnis 4 zu 1 zu Gunsten von AlphaGo. In Anerkennung des Sieges wurde AlphaGo von der Korea Baduk Association mit einem Ehren-9-Dan ausgezeichnet \cite{AlphaGoDan}. Der Vorsprung und das Herausforderungsspiel mit Lee Sedol wurden in einem Dokumentarfilm mit dem Titel AlphaGo \cite{AlphaGoFilm} unter der Regie von Greg Kohs dokumentiert. Er wurde am 22. Dezember 2016 von Science als einer der zweiten Durchbruch des Jahres gewählt \cite{Science2016}.

AlphaGo und seine Nachfolger verwenden einen Monte Carlo Baumsuch-Algorithmus, um seine Züge auf der Grundlage von Wissen zu finden, das zuvor durch maschinelles Lernen \glqq gelernt \grqq{} wurde, insbesondere durch ein künstliches neuronales Netz (eine deep learning Methode) durch ausführliches Training, sowohl durch menschliches als auch durch Computerspiel \cite{Silver_2016}. Ein neuronales Netz wird trainiert, um AlphaGos eigene Zugauswahlen und auch die Partien der Gewinner vorherzusagen. Dieses neuronale Netz verbessert die Stärke der Baumsuche, was zu einer höheren Qualität der Zugauswahl und einem stärkeren Selbstspiel in der nächsten Iteration führt.

\textbf{AlphaGo Spielstil}

Im Spiel gegen einen Top-Go-Spieler, hat das künstliche Intelligenzprogramm AlphaGo die Kommentatoren mit Zügen verwirrt, die oft als \glqq schön\grqq{} beschrieben werden, aber nicht in den üblichen menschlichen Spielstil passen \cite{Ribeiro2016}.


Howard Yu,Professor für strategisches Management und Innovation an der IMD Business School meinte, dass AlphaGo  eine Maschine darstellt, die nicht nur denkt, sondern auch lernen und Strategien entwickeln kann \cite{Ribeiro2016}.

Toby Manning, der Match-Schiedsrichter für AlphaGo vs. Fan Hui, hat den Stil des Programms als \glqq konservativ \grqq{} beschrieben \cite{Gibney2016}. Der Spielstil von AlphaGo begünstigt stark die größere Wahrscheinlichkeit, mit weniger Punkten zu gewinnen, gegenüber der geringeren Wahrscheinlichkeit, mit mehr Punkten zu gewinnen \cite{Ribeiro2016}. Seine Strategie, die Gewinnwahrscheinlichkeit zu maximieren, unterscheidet sich von dem, wozu menschliche Spieler neigen, nämlich territoriale Gewinne zu maximieren und erklärt einige seiner seltsam aussehenden Züge \cite{Chouard2016}.

