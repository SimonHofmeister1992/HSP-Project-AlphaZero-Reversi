% [12pt,a4paper,bibliography=totocnumbered,listof=totocnumbered]{scrartcl}
\documentclass[12pt,a4paper]{article}
\usepackage{mathtools}
\input{includes}

\begin{document}
\input{commands}


% ----------------------------------------------------------------------------------------------------------
% Titelseite
% ----------------------------------------------------------------------------------------------------------
\MyTitlepage{}{
\texttt{simon1.hofmeister@st.oth-regensburg.de}\\
\texttt{nadiia1.matsko@st.oth-regensburg.de}\\
\texttt{monika.silber@st.oth-regensburg.de}\\
\texttt{simon.wasserburger@st.oth-regensburg.de}}
{15.03.\the\year}

\setcounter{page}{1} 
% ----------------------------------------------------------------------------------------------------------
% Inhaltsverzeichnis
% ----------------------------------------------------------------------------------------------------------
\tableofcontents
\pagebreak


% ----------------------------------------------------------------------------------------------------------
% Inhalt
% ----------------------------------------------------------------------------------------------------------
% Abstände Überschrift
\titlespacing{\section}{0pt}{12pt plus 4pt minus 2pt}{-6pt plus 2pt minus 2pt}
\titlespacing{\subsection}{0pt}{12pt plus 4pt minus 2pt}{-6pt plus 2pt minus 2pt}
\titlespacing{\subsubsection}{0pt}{12pt plus 4pt minus 2pt}{-6pt plus 2pt minus 2pt}

% Kopfzeile
\renewcommand{\sectionmark}[1]{\markright{#1}}
\renewcommand{\subsectionmark}[1]{}
\renewcommand{\subsubsectionmark}[1]{}
\lhead{Kapitel \thesection}
\rhead{\rightmark}

\onehalfspacing
\renewcommand{\thesection}{\arabic{section}}
\renewcommand{\theHsection}{\arabic{section}}
\setcounter{section}{0}
\pagenumbering{arabic}
\setcounter{page}{1}

% ----------------------------------------------------------------------------------
% Kapitel: Einleitung
% ----------------------------------------------------------------------------------

\section{Einleitung}
\subsection{Reversi}
\subsection{ ReversiXT}
\subsection{AlphaGo}
\subsection{AlphaZero} 


\newpage

% ----------------------------------------------------------------------------------
% Kapitel: Related Work/Theorie
% ----------------------------------------------------------------------------------
\section{Related Work}

\subsection{Monte Carlo Tree Search}
Der MCTS findet als Alternative zum Alpha-Beta-Search Anwendung. Da der Alpha-Beta-Ansatz einen geringen Verzweigungsgrad und eine angemessene Bewertungsfunktion fordert, ist dessen in Anwendung in Brettspielen, die diese Bedingungen nicht erfüllen, ungeeignet. Der MCTS hingegen bewies sich im Umgang mit solchen Situation \cite{Chaslot2008}.
Basierend auf einer Bestensuche und stochastischer Simulation  wählt der MCTS bei jedem Durchlauf den erfolgversprechendsten Knoten als nächsten Spielzug aus. Entsprechend wird ein Baum aufgebaut, in dem ein Knoten einen konkreten Spielzustand widerspiegelt\cite{Chaslot2008}. Dabei werden die drei Schritte Expansion, Simulation und Backpropagation durchgeführt \cite{Chaslot2008}. Falls der aktuelle Spielzustand noch nicht als Knoten existiert, wird der Baum zunächst expandiert. Um nun die beste Aktion zu ermitteln, werden Spiele ausgehend vom aktuellen Zustand bis hin zum Spielende simuliert. Dabei werden valide Spielzüge zufällig ausgewählt. Hierbei ist jedoch zu beachten, dass eine reine Zufallsauswahl, die impliziert, dass die Selektion aller Möglichkeiten gleich wahrscheinlich ist, in eher primitivem Spielverhalten resultiert. Mithilfe einer Heuristik können daher aussichtsreichere Spielzüge favorisiert werden. Innerhalb eines Playouts durchlaufene Nodes werden schließlich aktualisiert, indem vermerkt wird, dass sie einmal mehr besucht wurden und welches Spielergebnis sich ergeben hat \cite{Chaslot2008}. 

Anzumerken ist, dass zwei verschiedene Policies genutzt werden. Für die Erweiterung des Baums wird eine Tree Policy angewendet, die besagt, dass entsprechende Blattknoten an bereits vorhandene, unbesuchte Knoten angefügt werden. Des Weiteren legt die Default Policy die Simulation fest. Hierbei wird in einem nichtterminalen Spielzustand, der gewöhnlich dem neu hinzugefügt Blattknoten entspricht, ein zufälliges Spiel durchlaufen, um ein ein Spielergebnis zu ermitteln \cite{Browne2012}. 

Der MCTS bleibt so lange aktiv, bis er unterbrochen wird, beispielsweise aufgrund von abgelaufener Rechenzeit. Der zu diesem Zeitpunkt als am erfolgreichsten ermittelte Knoten beziehungsweise Spielzug steht daraufhin fest \cite{Browne2012}.

Zu einem Knoten gehören der entsprechende Spielzustand, den er widerspiegelt, der Spielzug aus dem er resultierte, sowie der aus Simulationen resultierende Reward und wie oft er besucht wurde \cite{Browne2012}.

Nachfolgend werden die konkreten Umsetzungsdetails des MCTS in AlphaGo Zero betrachtet, so wie sie von Silver et al. angewendet wurden \cite{Silver2017}.
In AlphaGo Zero besteht der Selection-Schritt daraus, ausgehend von einem Wurzelknoten eine Simulationen bis hin zu einem Blattknoten zu durchlaufen. Dafür wird in jedem Schritt der Folgeknoten ausgewählt, der den maximalen UCT-Wert enthält \cite{Silver2017}.

Entsprechend dem AlphaZero-Ansatz findet bei der Simulation keine zufällige Auswahl des Spielzuges statt, stattdessen wird eine Variante des Upper Confidence Bound (UCB) angewendet. Die konkrete Abwandlung ist der polynomiale UCB applied to Trees (UCT). Dieser errechnet sie wie folgt:

\begin{equation}
UCT = \frac{Q(n_i)}{N(n_i)} cP(n) \frac{\sqrt{N(n)}}{1+N(n_i)}
\end{equation} 
Wobei c eine Konstante darstellt, die das Maß an Exploration festlegt und P(s,a) für die a-priori-Wahrscheinlichkeit für die Auswahl des jeweiligen Spielzugs steht. Letztlich wird stets der Spielzug ausgewählt, der den maximalen UCT-Wert darstellt \cite{Silver2017}.

Ferner ist an dieser Stelle anzumerken, dass der UCB den Kompromiss zwischen Exploration und Exploitation widerspiegelt. Der erste Quotient der UCT-Gleichung steht für das Maß der Ausbeutung und lenkt den Algorithmus dahingehend vielversprechende Knoten weiter zu besuchen. Im Gegensatz dazu wird dazu angehalten Bereiche, die noch nicht oft aufgesucht wurden, verstärkt zu untersuchen. Dies wird als Erkundung bezeichnet und durch den letzten Term des UCT abgebildet. Wichtig hierbei ist es ein Gleichgewicht der beiden Komponenten zu finden \cite{Browne2012}.

Die klassische Simulation von zufälligen Spieldurchläufen entfällt im MCTS vollständig, da noch nicht expandierte Knoten an das NN weitergegeben und dort evaluiert wird. Sobald ein solcher Blattknoten erreicht ist, wird dieser dem NN übergeben, woraufhin die a-priori-Wahrscheinlichkeit und die Bewertung des Spielzugs ermittelt werden. Daraufhin ist der Blattknoten expandiert und alle ausgehenden Kanten beziehungsweise Kinderknoten werden mit initialen Werten belegt.
Anschließend erfolgt die Backpropagation, indem in allen durchlaufenen Knoten die Gewinnwahrscheinlichkeit aktualisiert, sowie die Anzahl der Besuche um den Wert Eins inkrementiert wird \cite{Silver2017}.

Letztlich wird ein konkreter Spielzug ausgehend vom Wurzelknoten selektiert. Dabei wird der Knoten, der am häufigsten besucht wurde als der beste Spielzug aufgefasst. Der ausgewählte Kindknoten wird der neue Wurzelknoten. Der von ihm ausgehend aufgebaute Teilbaum wird beibehalten, während der der restliche Baum verworfen wird \cite{Silver2017}.

\newpage
\subsection{Neuronales Netz}
\subsection{Reinforcement Learning}

\subsection{Zusammenspiel von MCTS und NN}
AlphaGoZero und AlphaZero basieren auf der Kombination des MCTS und des NN. Dabei gibt es zwei Schnittstellen zwischen diesen beiden Komponenten. 
Erstere liegt in der bereits erwähnten Bewertung von Blattknoten. Der MCTS durchläuft keinen simulierten Spielablauf, sondern überlässt die Evaluation des Knotens dem NN und arbeitet mit den zurückgegeben Daten weiter. Dies spiegelt die policy evaluation wider \cite{Silver2017}.

%TODO reinforcement learning def of policy evaluation and improvement

Eine weitere Verzweigung von MCTS und NN tritt bei dem Update der Parameter des NN auf. Das NN ermittelt zu jedem Spielstand die möglichen Züge und gibt deren Gewinnwahrscheinlichkeit, sowie die Wahrscheinlichkeit für die Auswahl des Zuges an. Für die Aktualisierung der Parameter werden die genannten Werte dahingehend angepasst, dass sie den im MCTS ermittelten Werten entsprechend beziehungsweise sich diesen annähern. Eine Anpassung in diese Richtung ist sinnvoll, da die Daten im MCTS als deutlich genauer gelten. Dieser Vorgang entspricht der policy improvement \cite{Silver2017}.

\newpage
% ----------------------------------------------------------------------------------
% Kapitel: Implementierung/Umsetzung
% ----------------------------------------------------------------------------------
\section{Implementierung}

\subsection{Monte Carlo Tree Search}
Um den MCTS für Reversi zu realisieren wurden die Klassen MCTS und Node angelegt. Letztere enhält zwei überladene Konstruktoren zum Anlegen von Wurzel- und Kindknoten. Ein Node enthält zusätzliche Attribute. Sowohl der Elternknoten als auch eine ArrayList vom Typ Node, die die direkten Kinder enthält, werden abgespeichert. Die Anzahl, wie oft ein Knoten besucht wurde, wird in der Variable \texttt{numVisited} hinterlegt. Außerdem gespeichert wird das Ergebnis eines simulierten Spieldurchlaufs in \texttt{simulationReward}. Da der aktuelle Spielzustand durch das Environment, das ebenfalls die derzeitige Repräsentation des Playgrounds beinhaltet, definiert ist, wird dies ebenfalls im Node hinterlegt. Des Weiteren wird in dem Attribut \texttt{nextPlayer} festgehalten, welcher Spieler als Nächstes an der Reihe ist.

Es gibt einen überladenen Konstruktor, der einerseits für das Anlegen einer neuen Wurzel und andererseits für das Erzeugen eines neuen Kinderknotens zuständig ist. Bei der Instanziierung durch die Konstruktoren werden sinnvolle initiale Werte vergeben. numVisited und simulationReward werden auf 0 beziehungsweise 0.0 gesetzt. Für den Wurzelknoten gilt, dass er keinen Parent besitzt, für alle weiteren Knoten wird der Parent übergeben und gesetzt. Die Children werden zunächst durch eine leere Liste initialisiert. Der \texttt{nextPlayer} wird ebenfalls übergeben und gesetzt. Außerdem zu erwähnen ist die Methode \texttt{calculateUCT()}, die den Upper Confidence Bound applied to Trees (UCT) für einen Knoten berechnet. Sie ermittelt zunächst die Exploitation-Komponente, indem sie den simulationReward durch die Anzahl an Besuchen dividiert. Die Exploration berechnet sich aus dem Verhältnis, wie oft der Parent besucht wurde, geteilt durch den inkrementierten Wert, wie oft der aktuelle Knoten besucht wurde. Aus dem Quotient wird anschließend die Wurzel gezogen. Um den Kompromiss zwischen diesen beiden Komponenten zu kontrollieren, wird die Exploration mit der A-priori-Wahrscheinlichkeit für einen Zug multipliziert. Diese wird im neuralen Netz trainiert.

Hinsichtlich der Klasse MCTS ist festzuhalten, dass diese in der Klasse Agent über den Konstruktoraufruf instanziiert wird. Dieser verlangt das Environment und den Player als Übergabeparameter und legt daraufhin einen neuen Wurzelknoten an, sowie eine leere ArrayList vom Typ Node, die die Blattknoten beinhaltet, die im späteren Verlauf simuliert werden müssen. Für die Simulation muss beachtet werden, dass das Environment geklont und somit eine tiefe Kopie erzeugt werden muss, damit der tatsächliche Spielzustand nicht unbeabsichtigt manipuliert wird. Dabei ist festzuhalten, dass dies zweimal stattfindet. Einmal, wenn ein neuer Baum aufgebaut wird, somit erhält der neue Wurzelknoten und ebenfalls jedes seiner Kinder jeweils einen eigenen Klon. Für die Kinderknoten gilt, dass diese ihre Environment-Instanz an ihre Kinder weitergeben und diese somit innerhalb derselben Instanz agieren. Um den MCTS zu starten, wird die Methode \texttt{searchBestTurn()} aufgerufen. Diese expandiert zunächst den Wurzelknoten, indem sie alle im aktuellen Zustand möglichen validen Züge ermittelt und durch diese iteriert. Die Methode \texttt{getPossibleTurns()} gibt diese zurück. Sie iteriert über das gesamte Spielfeld und prüft dabei mithilfe der Methode \texttt{validateTurnPhase1()} im Environment, an welcher Stelle ein gültiger Zug gemacht werden kann. Für jeden dieser Züge wird in \texttt{expand()} der Kinderknoten angelegt, sowie als unbesuchter Blattknoten abspeichert. Außerdem wird ermittelt, welcher Spieler als Nächster einen Zug machen darf und der simulierte derzeitige Spielzustand anhand des Zuges aktualisiert. Nachdem ein Knoten expandiert wurde, wird er aus der Liste der Blattknoten wieder entfernt.
Daraufhin werden die unbesuchten Blattknoten, die am Anfang den Kinderknoten der Wurzel entsprechen, in der Methode \texttt{traverse()} durchlaufen. Dabei wird in jedem dieser eine Simulation gestartet, die einen Spielverlauf bis zum Spielende anhand zufällig ausgewählter möglicher Züge durchspielt. 

Mithilfe der Funktion \texttt{simulate()} erfolgt die Simulation eines Spiels. Ein Zufallszug wird durch eine Instanz der Java-Klasse Random generiert. Hierbei wird ein zufälliger Integer erzeugt, der durch eine Modulo-Operation auf den Größenbereich abgestimmt wird, der dem der Anzahl der möglichen Züge entspricht. Die resultierende Zahl gibt den auszuwählenden Zug innerhalb der ArrayList an.

Wenn keine weiteren validen Folgezüge ermittelt werden können, bedeutet das das Spielende und der Reward für die Spielausgang wird anhand der Funktion \texttt{rewardGameState()} berechnet. Diese erhält als Parameter das Environment, sowie den Spieler, für den der Reward kalkuliert werden soll. Indem der gesamte Playground durchlaufen und gezählt wird, wie viele Steine vom übergebenen Spieler enthalten sind, errechnet sich die Bewertung des Spiels. Abschließend werden die Werte für Anzahl Besuche und Reward ebenfalls im Wurzelknoten aktualisiert.

Nach Abschluss der Simulation wird der Reward zurückgegeben. Daraufhin wird in \texttt{traverse()} die Backpropagation der Ergebnisse durchgeführt, indem iterativ vom aktuellen Knoten bis hoch zur Wurzel die Anzahl an Besuchen inkrementiert und der Reward entsprechend erhöht wird.

\newpage

% ----------------------------------------------------------------------------------
% Kapitel: Allgemeine Informationen/Organisation
% ----------------------------------------------------------------------------------
\section{Organisation}

\subsection{Team und Aufgabenverteilung}
% Beschreiben Sie in diesem Abschnitt Ihr Team. Welche Person hat welche Aufgaben wahrgenommen, wie wurden
% Aufgaben aufgeteilt und wie wurde kommuniziert, etc.

\subsection{Kommunikation}

\subsection{Versionskontrolle}

\subsection{OS, IDE und Programmiersprache}

\subsection{Testumgebungen}

\subsection{Projekt-Dokumentation}

\newpage
% ----------------------------------------------------------------------------------
% Kapitel: Fazit
% ----------------------------------------------------------------------------------
\section{Fazit}
% Beschreiben Sie in diesem Abschnitt u.a.\ was Ihnen an diesem Fach gefallen hat und welche
% Verbesserungsvorschläge Sie für künftige Veranstaltungen haben. Was konnten Sie dazulernen, in welchen
% Bereichen haben Sie sich verbessert. Welche Problemsituationen gab es während der Projekterstellung, wie
% sind Sie diese angegangen und wie haben Sie diese gelöst. Was haben Sie evtl.\ vermisst.

\pagebreak

%\newpage
% ----------------------------------------------------------------------------------
% Kleine Einführung in LaTeX-Elemente
% ----------------------------------------------------------------------------------
% \section{\LaTeX-Elemente}
% Dieser Abschnitt soll nicht Bestandteil des Projektberichtes sein, sondern beinhaltet lediglich einige
% Informationen über \LaTeX-Distributionen, Editoren und \LaTeX-Elemente, die Ihnen beim Einstieg in das
% \LaTeX-Textsatzsystem helfen sollen.

% \subsection{\LaTeX-Distributionen nach Betriebssystemen}

% \subsubsection{\LaTeX-Distributionen}

% Folgende Haupt-\LaTeX-Distributionen stehen Ihnen zur Verfügung:
% \begin{itemize}
%  \item Windows:\quad \texttt{MiKTeX}\quad Webseite:\quad\url{http://www.miktex.org}
%  \item Linux/Unix:\quad \texttt{TeX Live}\quad Webseite:\quad\url{http://tug.org/texlive/}
%  \item Mac OS:\quad \texttt{MacTeX}\quad Webseite:\quad\url{http://www.tug.org/mactex/}
% \end{itemize}

% \subsubsection{\LaTeX-Editoren}
% Auf folgenden Webseiten können Sie einige hilfreiche \LaTeX-Editoren finden:
% \begin{itemize}
%  \item Windows/Linux/Mac OS: \url{http://www.xm1math.net/texmaker/}
%  \item Windiws: \url{http://www.texniccenter.org/}
%  \item Mac OS: \url{http://pages.uoregon.edu/koch/texshop/}
% \end{itemize}

%Falls bei den oben genannten Editoren kein passender vorhanden war, findet sich auf Wikipedia eine  %Zusammenstellung vieler weiterer \LaTeX-Editoren:\\[1em]
%\hspace*{3cm}\url{https://en.wikipedia.org/wiki/Comparison_of_TeX_editors}


%\subsection{Unterabschnitt}
%Zum Einfügen eines Bildes, siehe Abbildung \ref{fig:reversi01}, wird die
% \textit{minipage}-Umgebung 
%genutzt, da die Bilder so gut positioniert werden können.

%\vspace{1em}
%\begin{minipage}{\linewidth}
%	\centering
%	\includegraphics[width=0.6\linewidth]{pics/gamefield01.png}
%	\captionof{figure}[Spielfeld 01]{Unbespieltes Spielfeld\footnotemark }
%	\label{fig:reversi01}
%\end{minipage}
%\footnotetext{Diesem Spielfeld wurden noch keine Spieler zugewiesen (daher die
% dunklen Spielsteine)}

%Nachdem das Spiel gestartet wurde und beiden Spielphasen durchlaufen wurden, siegt
% schließlich der %Spieler mit der Farbe rot.

%\vspace{1em}
%\begin{minipage}{\linewidth}
%	\centering
%	\includegraphics[width=0.6\linewidth]{pics/gamefield02.png}
%	\captionof{figure}[Spielfeld 02]{Finales Spielfeld\footnotemark }
%	\label{fig:reversi2}
%\end{minipage}
%\footnotetext{Das Spielfeld nach der Zug- und Bombenphase. Spieler rot gewinnt
% eindeutig.}

%\subsection{Tabellen}
%In diesem Abschnitt wird eine Tabelle (siehe Tabelle \ref{tab:beispiel}) dargestellt.

%\vspace{1em}
%\begin{table}[!h]
%	\centering
%	\begin{tabular}{|l|l|l|}
%		\hline
%		\textbf{Name} & \textbf{Name} & \textbf{Name}\\
%		\hline
%		1 & 2 & 3\\
%		\hline
%		4 & 5 & 6\\
%		\hline
%		7 & 8 & 9\\
%		\hline
%	\end{tabular}
%	\caption{Beispieltabelle}
%	\label{tab:beispiel}
%\end{table}


%\subsection{Auflistung}
%Für Auflistungen wird die \textit{enumerate}- oder \textit{itemize}-Umgebung genutzt.
%
%\begin{itemize}
%	\item Nur
%	\item ein
%	\item Beispiel.
%\end{itemize}
%
%\subsection{Listings}
%Zuletzt ein Beispiel für ein Listing, in dem Quellcode eingebunden werden kann, siehe Listing \ref{lst:arduino}.
%
%\vspace{1em}
%\begin{lstlisting}[caption=Arduino Beispielprogramm, label=lst:arduino]
%int ledPin = 13;
%void setup() {
%    pinMode(ledPin, OUTPUT);
%}
%void loop() {
%    digitalWrite(ledPin, HIGH);
%    delay(500);
%    digitalWrite(ledPin, LOW);
%    delay(500);
%}
%\end{lstlisting}
%
%\subsection{Tipps}
%Die Quellen befinden sich in der Datei \textit{quellen.bib}. Eine Buch- und eine Online-Quelle sind beispielhaft eingefügt. [Vgl. \cite{buch}, \cite{online}]
%
%\pagebreak

% ----------------------------------------------------------------------------------------------------------
% Literatur
% ----------------------------------------------------------------------------------------------------------
\renewcommand\refname{Literatur}
\bibliographystyle{alpha}
\bibliography{literatur}

%\printbibliography

\pagebreak

% ----------------------------------------------------------------------------------------------------------
% Anhang
% ----------------------------------------------------------------------------------------------------------
\pagenumbering{Roman}
\setcounter{page}{1}
%\lhead{Anhang \thesection}

\begin{appendix}
\section*{Anhang}
%\phantomsection
\addcontentsline{toc}{section}{Anhang}
\addtocontents{toc}{\vspace{-0.5em}}

\end{appendix}

\end{document}